
Настоящая лабораторная работа является продолжением работы № 2 по дисциплине <<Методы и алгоритмы обработки и анализа изображений>>.

Выполнение лабораторной работы предполагает создание классификатора, обучаемого с учителем и осуществляющего распознавание символов на изображениях. В рамках лабораторной работы необходимо разработать программу, выполняющую следующие действия:

\begin{itemize}

	\item обучение каждого из трех, описанных в методических указаниях, классификаторов на наборе векторов признаков, сгенерированных для образов символов, полученных с помощью метода \verb|operator(unsigned)| класса \verb|CCaptcha|;
	\item тестирование качества обучения каждого из классификаторов до достижения объема правильно классифицированных векторов тестовой выборки не менее 95 \% от объема выборки;
	\item тестирование каждого из классификаторов на наборе (не менее 1000) изображений, сгенерированных с помощью метода \verb|operator()| класса \verb|CCaptcha|, с подсчетом правильно распознанных изображений в целом и правильно классифицированных символов на изображениях в частности.

	Цель тестирования - построение классификатора, дающего правильное распознавание для 85 \% отдельных символов и 75 \% изображений.

\end{itemize}

Рекомендуется включать в состав векторов признаков не только Хью - моменты, но и прочие моменты, рассчитываемые функцией \verb|moments()| библиотеки OpenCV.

По результатам выполнения лабораторной работы необходимо сделать вывод о том, какой из классификаторов показал лучший результат и почему.

