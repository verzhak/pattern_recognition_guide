
В рамках настоящей лабораторной работы решается задача расчета количества букв в некотором неизвестном алфавите по тексту, буквы в котором распределены равномерно. Всего имеется шесть алфавитов, поэтому для выполнения лабораторной работы группа должна разбиться на шесть бригад, каждая из которых получит уникальный алфавит для анализа.

Каждый текст представляет собой одноканальное 8-ми битное беззнаковое изображение, буквы на котором расположены в окнах 200 на 200 пикселей, начиная с пикселя с координатами $(0, 0)$.

Расчет количества букв необходимо реализовать программно на языке программирования C++ с использованием библиотеки OpenCV. Расчет должен состоять из следующих этапов:

\begin{enumerate}

	\item загрузка исходного изображения;
	\item формирование массива дескрипторов окон 200 на 200 пикселей;
	\item классификация массива дескрипторов окон с помощью алгоритма k-means;
	\item удаление пустых классов;
	\item объединение классов, для которых выполняется условие близости.

	Два класса подлежат объединению, если Эвклидово расстояние между их центрами не превосходит заданного порога:

	\begin{itemize}

		\item $0.003$ (бригада № 1);
		\item $0.003$ (бригада № 2);
		\item $0.004$ (бригада № 3);
		\item $0.005$ (бригада № 4);
		\item $0.004$ (бригада № 5);
		\item $0.003$ (бригада № 6);

	\end{itemize}

	\item удаление классов, количество векторов в которых не превосходит пяти;
	\item подсчет количества непустых классов;
	\item сравнение количества непустых классов с количеством букв в алфавите (преподавателю допускается давать подсказки <<больше>> или <<меньше>>).

\end{enumerate}

Пример расчета количества букв приведен на рисунках \ref{image:1:src} и \ref{image:1:res}.

На рисунке \ref{image:1:src} приведено изображение исходного текста.
На рисунке \ref{image:1:res} приведено результирующее изображение - буквы одного класса окрашены в одинаковый цвет (всего - 31 буква, используется часть алфавита Кирт, изобретенного Дж. Р. Р. Толкином).

\mimage{1:src}{src_scale}{Исходный текст}{height = 0.95\textheight}
\mimage{1:res}{res_scale}{Результирующее изображение}{height = 0.95\textheight}

