
В рамках настоящей лабораторной работы требуется разработать и реализовать алгоритм, выделяющий на исходном изображении несколько символов с целью их дальнейшего распознавания в следующих лабораторных работах.

Для генерации исходных изображений необходимо использовать метод \verb|operator()| класса \verb|CCaptcha|, библиотека с реализацией которого распространяется в качестве приложения к методическим указаниям по лабораторной работе.

Для выделения символов необходимо использовать пространственную фильтрацию, пороговое преобразование, арифметические действия над изображениями и действия над контурами.

Результатом лабораторной работы, достаточным для ее сдачи, является программа, выполняющая для 85 - 90 \% сгенерированных изображений качественное выделение символов. Под <<качественным выделением>> здесь понимается:

\begin{itemize}

	\item бинарность изображения (фон - черный, образ символа - белый);
	\item минимальность помех на изображении;
	\item связность частей образов символов;
	\item гладкость краев образов символов.

\end{itemize}

Для каждого исходного изображения программа должна также формировать новое изображение, на котором будут отрисованы наибольшие по площади контуры, составляющие каждый символ. Означенные контуры должны быть аппроксимированы с целью устранения <<шероховатостей>> на их границах.

