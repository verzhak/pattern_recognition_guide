
В рамках настоящей лабораторной работы предлагается устранить периодические помехи на исходной видеопоследовательности, объединив улучшенные кадры в новую видеопоследовательность.

Периодические помехи, наличествующие на исходной видеопоследовательности, характеризуются круговым частотным спектром, поэтому для их минимизации может быть использован идеальный режекторный фильтр, задача подбора параметров которого суть есть ключевая задача лабораторной работы.

В рамках настоящей лабораторной работы требуется разработать следующие программы:

\begin{itemize}

	\item программу, выполняющую частотную фильтрацию исходной видеопоследовательности и сохраняющую результат фильтрации в виде новой видепопоследовательности;
	\item программу, выводящую на экран исходную видеопоследовательность и видеопоследовательность - результат фильтрации.

\end{itemize}

При выводе видеопоследовательности на экран необходимо помнить, что частота кадров в ней\footnote{Справедливо для задания каждой из бригад.} равна 25-ти FPS.

