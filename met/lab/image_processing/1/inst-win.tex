
Установка библиотеки OpenCV в ОС семейства Windows производится с помощью следующих дистрибутивов библиотеки:

\begin{itemize}

	\item дистрибутив, распространяемый преподавателем - для использования библиотеки в проектах, разрабатываемых в IDE Visual Studio 2005;
	\item дистрибутив, доступный для бесплатной загрузки с оффициального сайта библиотеки \cite{opencv} - для использования библиотеки в проектах, разрабатываемых в IDE Visual Studio версий 2008 и 2010.

\end{itemize}

Для подключения библиотеки OpenCV версии 2.3.1\footnote{Инструкция относится к дистрибутиву библиотеки OpenCV, распространяемому преподавателем.} к проекту, разрабатываемому на языке программирования C++ с помощью интегрированной среды разработки (Integrated Developm\-ent Environment; IDE) Visual Studio версии 2005, необходимо выполнить следующие действия:

\begin{enumerate}

	\item добавить в переменную <<Path>> окружения путь к подкаталогу bin каталога установки библиотеки\footnote{В каталоге bin расположены файлы динамических библиотек библиотеки OpenCV - ОС должна знать, где файлы динамических библиотек можно найти.}, для чего необходимо выполнить следующие действия:

	\begin{enumerate}

		\item открыть окно <<Панель управления>>;
		\item открыть окно <<Свойства системы>>;
		\item в окне <<Свойства системы>> выбрать вкладку <<Дополнительно>> (рисунок \ref{image:1:101}), после чего необходимо нажать кнопку <<Переменные среды>>;
		\item в поле <<Системные переменные>> окна <<Переменные среды>> выбрать переменную <<Path>> (рисунок \ref{image:1:102}), после чего необходимо нажать кнопку <<Изменить>>;
		\item в окне <<Изменение системной переменной>> с помощью поля <<Значение переменной>> дописать в конец переменной <<Path>> через точку с запятой полный путь до подкаталога bin каталога, в который была установлена библиотека (рисунок \ref{image:1:103}), после чего необходимо нажать кнопку <<OK>>;

	\end{enumerate}

	\item запустить на выполнение IDE Visual Studio 2005;
	\item открыть в IDE проект, к которому требуется подключить библиотеку OpenCV;
	\item открыть окно свойств проекта;
	\item подключить библиотеку к отладочной конфигурации проекта:

	\begin{enumerate}

		\item в поле <<Configuration>> окна свойств проекта выбрать пункт <<Debug>> (рисунок \ref{image:1:201});
		\item в поле <<Additional Include Directories>> вкладки <<Configuration Properties / C, C++ / General>> указать полный путь до каталога, в котором расположены заголовочные файлы библиотеки OpenCV (данным каталогом является подкаталог include каталога, в который была выполнена установка библиотеки; рисунок \ref{image:1:202});
		\item в поле <<Additional Library Directories>> вкладки <<Configuration Properties / Linker / General>> (рисунок \ref{image:1:203}) указать через точку с запятой полные пути до следующих каталогов:

		\begin{itemize}

			\item каталог, содержащий файлы динамических библиотек библиотеки OpenCV (данным каталогом является подкаталог bin каталога, в который была выполнена установка библиотеки);
			\item каталог, содержащий файлы статических библиотек библиотеки OpenCV, необходимые для связывания проекта с соответствующими динамическими библиотеками (данным каталогом является подкаталог lib каталога, в который была выполнена установка библиотеки);

		\end{itemize}

		\item В поле <<Additional Dependencies>> вкладки <<Configuration Properties / Linker / Input>> перечислить через перевод строки имена следующих файлов:
		
		\begin{itemize}
		
			\item opencv\_calib3d231d.lib;
			\item opencv\_contrib231d.lib;
			\item opencv\_core231d.lib;
			\item opencv\_features2d231d.lib;
			\item opencv\_flann231d.lib;
			\item opencv\_highgui231d.lib;
			\item opencv\_imgproc231d.lib;
			\item opencv\_legacy231d.lib;
			\item opencv\_ml231d.lib;
			\item opencv\_objdetect231d.lib;
			\item opencv\_ts231d.lib;
			\item opencv\_video231d.lib.
			
		\end{itemize}
		
		Таковое перечисление является указанием компилятору связать исполняемый файл проекта с динамическими библиотеками библиотеки OpenCV (рисунки \ref{image:1:204} и \ref{image:1:205});

	\end{enumerate}

	\item подключить библиотеку к release - конфигурации проекта.

	Для подключения библиотеки к release - конфигурации проекта необходимо выполнить те же действия, что и для отладочной конфигурации проекта, с тем лишь исключением, что в поле <<Additional Dependencies>> вкладки <<Configuration Properties / Linker / Input>> необходимо перечислить имена других файлов (рисунок \ref{image:1:207}):

	\begin{itemize}
		
		\item opencv\_calib3d231.lib;
		\item opencv\_contrib231.lib;
		\item opencv\_core231.lib;
		\item opencv\_features2d231.lib;
		\item opencv\_flann231.lib;
		\item opencv\_highgui231.lib;
		\item opencv\_imgproc231.lib;
		\item opencv\_legacy231.lib;
		\item opencv\_ml231.lib;
		\item opencv\_objdetect231.lib;
		\item opencv\_ts231.lib;
		\item opencv\_video231.lib.
			
	\end{itemize}

\end{enumerate}

\mimage{1:101}{win/101}{Окно <<Свойства системы>> - вкладка <<Дополнительно>>}{width = \linewidth}
\mimage{1:102}{win/102}{Окно <<Переменные среды>>}{}
\mimage{1:103}{win/103}{Окно <<Изменение системной переменной>>}{}
\begin{landscape}
\mimage{1:201}{win/201}{Поле <<Configuration>>}{width = 0.94\linewidth}
\mimage{1:202}{win/202}{Поле <<Additional Include Directories>>}{width = 0.94\linewidth}
\mimage{1:203}{win/203}{Поле <<Additional Library Directories>>}{width = 0.94\linewidth}
\mimage{1:204}{win/204}{Поле <<Additional Dependencies>>}{width = 0.94\linewidth}
\end{landscape}
\mimage{1:205}{win/205}{Редактирование поля <<Additional Dependencies>> (отладочная конфигурация проекта)}{}
\mimage{1:207}{win/207}{Редактирование поля <<Additional Dependencies>> (release - конфигурация проекта)}{}

