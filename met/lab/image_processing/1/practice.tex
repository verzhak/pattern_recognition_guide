
Для ознакомления с основами написания программ, использующих функционал программной библиотеки OpenCV, каждой из бригад, сформированных из студентов группы,\footnote{По 2 - 3 человека в бригаде.} необходимо разработать интерпретатор языка программирования Brainloller \cite{brainloller}, способный выполнить набор контрольных программ, выданных преподавателем.

Программа на языке программирования Brainloller представляет из себя изображение, каждая связная квадратная область\footnote{В настоящей лабораторной работе используются области рамером 20 на 20 пикселей.} которого, заполненная одним цветом, кодирует одну инструкцию. После выполнения очередной инструкции интерпретатор языка программирования Brainloller переходит к ячейке в соответствии с текущим направлением перехода (в начале программы текущее направление переход устанавливается в переход на одну инструкцию вправо).

Программа оперирует бесконечно большой лентой памяти\footnote{Для практических реализаций интерпретатора языка программирования Brainloller длина ленты памяти предполагается конечной.}. В начале выполнения программы указатель на текущую ячейку указывает на первую ячейку ленты памяти. Указатель может перемещаться вправо и влево по ленте памяти, кроме случая, когда указатель указывает на первую ячейку ленты памяти - в данном случае перемещение влево невозможно. Все ячейки ленты памяти хранят целые числа; в начале выполнения программы в ячейки ленты памяти загружены нули. В некоторых случаях значения ячеек ленты памяти могут рассматриваться как коды ASCII - символов.

Для кодирования инструкций используются следующие цвета:

\newcommand{\blcolor}[5]{\item красный: 0x#3, зеленый: 0x#2, синий: 0x#1 - #4#5}

\begin{itemize}

	\blcolor{00}{FF}{00}{увеличить значение текущей ячейки ленты памяти на единицу}{;}
	\blcolor{00}{80}{00}{уменьшить значение текущей ячейки ленты памяти на единицу}{;} 
	\blcolor{FF}{00}{00}{сдвинуть указатель текущей ячейки ленты памяти вправо на одну ячейку}{;} 
	\blcolor{80}{00}{00}{сдвинуть указатель текущей ячейки ленты памяти влево на одну ячейку}{;} 
	\blcolor{FF}{FF}{00}{если значение текущей ячейки ленты памяти меньше или равно нулю, то поместить указатель инструкций на парную инструкцию 0x008080 по текущему направлению перехода}{;} 
	\blcolor{80}{80}{00}{переход на парную инструкцию 0xFFFF00 по направлению перехода, повернутому на $180^\circ$ относительно текущего направления перехода}{;}
	\blcolor{00}{00}{FF}{вывести на экран ASCII - символ с кодом, равным значению текущей ячейки ленты памяти}{;} 
	\blcolor{00}{00}{80}{считать с клавиатуры код ASCII - символа и поместить код в текущую ячейку памяти}{;} 
	\blcolor{00}{FF}{FF}{текущее направление перехода поворачивается на $90^\circ$ против часовой стрелки}{;}
	\blcolor{00}{80}{80}{текущее направление перехода поворачивается на $90^\circ$ по часовой стрелке}{.}

\end{itemize}

