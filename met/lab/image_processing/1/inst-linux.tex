
В ОС GNU / Linux для связывания исполняемого файла программы с библиотекой OpenCV компилятору необходимо передать аргументы, возвращаемые командой \linebreak \verb|pkg-config --cflags --libs opencv|

Если для разработки программы используется интегрированная среда разработки \linebreak (Integrated development environment; IDE) Code::Blocks, то для подключения библиотеки к проекту необходимо выполнить следующие действия:

\begin{enumerate}

	\item в контекстном меню проекта, вызываемом нажатием правой кнопки мыши по имени проекта во вкладке <<Projects>> окна <<Management>>, выбрать пункт <<Properties>> (рисунок \ref{image:1:1}), в результате чего будет открыто окно свойств проекта;
	\item в окне свойств проекта выбрать вкладку <<Libraries>>, в поле <<Filter>> которой необходимо ввести строку <<opencv>> (рисунок \ref{image:1:2}), после чего нужно нажать клавишу <<Enter>>;
	\item в дереве <<Available in pkg-config>> поля <<Known libraries>> вкладки <<Libraries>> выбрать библиотеку <<opencv>> (рисунок \ref{image:1:3});
	\item добавить библиотеку в число библиотек, с которыми будет связан проект, для чего необходимо нажать кнопку $<$ вкладки <<Libraries>> (рисунок \ref{image:1:4});
	\item закрыть окно свойств проекта нажатием кнопки <<OK>>.

\end{enumerate}

\begin{landscape}
\mimage{1:1}{1}{Открытие окна свойств проекта}{width = 0.9\linewidth}
\mimage{1:2}{2}{Поиск библиотеки OpenCV в базе библиотек утилиты pkg-config}{width = 0.9\linewidth}
\mimage{1:3}{3}{Результаты поиска в базе библиотек утилиты pkg-config}{width = 0.9\linewidth}
\mimage{1:4}{4}{Связывание проекта с библиотекой OpenCV}{width = 0.9\linewidth}
\end{landscape}

