
Везде в настоящем разделе буквосочетание POSTFIX необходимо заменять на следующие букво- и цифросочетания:

\begin{itemize}

	\item ничего - в случае, если дистрибутив программного продукта содержит версию продукта, собранную для ОС GNU / Linux (ia32);
	\item \_x86\_64 - в случае, если дистрибутив программного продукта содержит версию продукта, собранную для ОС GNU / Linux (x86\_64);
	\item \_vs\_8 - в случае, если дистрибутив программного продукта содержит версию продукта, собранную для Visual Studio 2005;
	\item \_vs\_9 - в случае, если дистрибутив программного продукта содержит версию продукта, собранную для Visual Studio 2008;
	\item \_vs\_10 - в случае, если дистрибутив программного продукта содержит версию продукта, собранную для Visual Studio 2010.

\end{itemize}

\mysubsection{Библиотека генерирования captcha}

Библиотека реализует класс \verb|CCaptcha| и классы \verb|CCaptchaBrigade1|, \verb|CCaptchaBrigade2| и \verb|CCaptchaBrigade3| - наследники класса \verb|CCaptcha|. Перечисленные классы позволяют генерировать captcha, проверять правильность введения captcha, генерировать изображения отдельных символов без помех, выводить последние сгенерированные изображения на экран.

Описание класса \verb|CCaptcha| приведено в листинге \ref{listing:adx:captcha:CCaptcha}.

\mylistingbegin{adx:captcha:CCaptcha}{Описание класса CCaptcha}
\begin{lstlisting}

class CCaptcha
{
	public:

		CCaptcha();
		~CCaptcha();

		Mat operator()();
		Mat operator()(unsigned num);

		void show();
		unsigned check(char * captcha);
		char alphabet(unsigned ind);
};

\end{lstlisting}
\mylistingend

Класс \verb|CCaptcha| содержит следующие методы:

\begin{itemize}

	\item оператор \verb|()| - генерирует новую captcha;
	\item оператор \verb|(unsigned num)| - генерирует изображение без помех символа с индексом \verb|num| из текущего алфавита (индексация ведется с нуля);
	\item метод \verb|show()| - выводит на экран последние сгенерированные изображения (captcha и последний сгенерированный символ);
	\item метод \verb|check()| - сравнивает введенную captcha с имеющейся и возвращает количество совпавших символов в captcha;
	\item метод \verb|alphabet()| - возвращает символ с указанным индексом из текущего алфавита.

\end{itemize}

Каждая из бригад использует вместо класса \verb|CCaptcha| один из его производных классов, отличающихся друг от друга используемыми алфавитами.

Кроме класса \verb|CCaptcha| библиотека экспортирует следующие константы типа \verb|unsigned|:

\begin{itemize}

	\item \verb|lwh| - количество строк и количество столбцов в изображении символа;
	\item \verb|nic| - количество символов в captcha;
	\item \verb|alphabet_size| - количество символов в алфавите.

\end{itemize}

Дистрибутив библиотеки состоит из следующих файлов, расположенных в каталоге \verb|captcha|:

\begin{itemize}

	\item captchaPOSTFIX.so - разделяемая библиотека (только в версиях для ОС GNU / Linux);
	\item captchaPOSTFIX.dll - разделяемая библиотека (только в vs - версиях);
	\item captchaPOSTFIX.lib - статическая библиотека для связывания разделяемой библиотеки с программой (только в vs - версиях);
	\item captcha.hpp - заголовочный файл библиотеки.

\end{itemize}

